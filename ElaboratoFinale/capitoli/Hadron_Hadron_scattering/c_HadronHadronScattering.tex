%Hadron-Hadron Scatteing
%%%%%-------PROVA-------------
\documentclass[10pt]{article}
\usepackage[british]{babel}
\usepackage{amsmath}
\usepackage{amsfonts}
\usepackage{amssymb}
\usepackage{geometry}
\geometry{hmargin={2cm,2cm},vmargin={2cm,2cm}}

\usepackage[colorlinks=true, linkcolor=blue]{hyperref} %to use autoref
%%%%%-------------------------

\begin{document}

\section{Hadron-Hadron Scattering}

In a proton-proton collision, in addition to the main hard scattering, also other scattering processes are possible. To well understood the proton-proton collision we need to describe the \textbf{Multi Parton Interaction} (MPI) and the \textbf{Beam Beam Remnants} (BBR) 

\subsection{Multiple Parton Interaction}

The aim of this section is to describe the basic concepts that are use to simulate MPI, for example in  \texttt{Pythia}.

\subsubsection{Basic Concepts}

The main hypothesis of the multiple interaction models is that the QCD factorization is true not only for the hard process but also for the other scatters.

So, we can write:

\begin{equation}
	\frac{d\sigma_{\text{int}}}{dp_\perp}=\displaystyle\sum_{i,j,k,l}\displaystyle\int dx_1 \displaystyle\int dx_2 \displaystyle\int d\hat{t}\, f_i(x_1,Q^2)f_j(x_2,Q^2)\frac{d\hat{\sigma}_{ij\,\rightarrow\,kl}}{d\hat{t}}\delta\left( p_\perp^2-\frac{\hat{t}\hat{u}}{\hat{s}} \right)
	\label{eq:sigma_int1}
\end{equation}


This is the differential cross section for QCD hard $2\ \rightarrow 2$ processes, this processes are the one reported in \mbox{Table \ref{table:partonic_cross_sections}}. In eq. (\ref{eq:sigma_int1}) we use the Mandelstam variables associated to the partonic system:

\begin{align}
	&\hat{s}=(p_1+p_2)^2=(p_3+p_4)^2=x_1x_2s\\
	&\hat{t}=(p_1-p_3)^2=(p_2-p_4)^2\\
	&\hat{u}=(p_1-p_4)^2=(p_2-p_3)^2
\end{align} 
wherer $p_1$, $p_2$ are the four-momenta of the incoming partons and $p_3$, $p_4$ the four-momenta of the outgoing partons. 
Note that in equation (\ref{eq:sigma_int1}) the jet cross section is twice as large $\sigma_{\ \text{jet}}=2\sigma_{\text{int}}$, because at first approximation each interaction give rise to two jets.

We assume also that the "hardness" of processes is defined by the $p_\perp$ scale of the interaction ($Q^2=p_\perp^2$).

\begin{table}[h!]
	\centering
	\begin{tabular}{l | c}\rule{0pt}{3ex} 
	Process & Partonic cross section \rule{0pt}{3.5ex}\\   \hline\hline 
	$q\,q'\ \rightarrow\ q\,q'$ & $\frac{4}{9}\frac{\hat{s}^2+\hat{u}^2}{\hat{t}^2}$\rule{0pt}{3.5ex}\\
	$q\,q\ \rightarrow\ q\,q$ & $\frac{4}{9}\left(\frac{\hat{s}^2+\hat{u}^2}{\hat{t}^2}+\frac{\hat{s}^2+\hat{t}^2}{\hat{u}^2}\right) -\frac{8}{27}\frac{\hat{s}^2}{\hat{u}\hat{t}}$\rule{0pt}{3.5ex}\\
	$q\,\overline{q}\ \rightarrow\ q'\,\overline{q}'$ & $\frac{4}{9}\frac{\hat{s}^2+\hat{u}^2}{\hat{t}^2}$\rule{0pt}{3.5ex}\\
	$q\,\overline{q}\ \rightarrow\ q\,\overline{q}$ & $\frac{4}{9}\left(\frac{\hat{s}^2+\hat{u}^2}{\hat{t}^2}+\frac{\hat{t}^2+\hat{u}^2}{\hat{s}^2}\right) -\frac{8}{27}\frac{\hat{u}^2}{\hat{s}\hat{t}}$\rule{0pt}{3.5ex}\\
	$q\,\overline{q}\ \rightarrow\ g\,g$ & $\frac{32}{27}\frac{\hat{t}^2+\hat{u}^2}{\hat{t}\hat{u}}-\frac{8}{3}\frac{\hat{t}^2+\hat{u}^2}{\hat{s}^2}$\rule{0pt}{3.5ex}\\
	$g\,g\ \rightarrow\ q\,\overline{q}$ & $\frac{1}{6}\frac{\hat{t}^2+\hat{u}^2}{\hat{t}\hat{u}}-\frac{8}{3}\frac{\hat{t}^2+\hat{u}^2}{\hat{s}^2}$\rule{0pt}{3.5ex}\\
	$g\,q\ \rightarrow\ g\,q$ & $-\frac{4}{9}\frac{\hat{s}^2+\hat{u}^2}{\hat{s}\hat{u}}-\frac{\hat{u}^2+\hat{s}^2}{\hat{t}^2}$\rule{0pt}{3.5ex}\\
	$g\,g\ \rightarrow\ g\,g$ & $\frac{9}{2}\left( 3-\frac{\hat{t}\hat{u}}{\hat{s}^2} -\frac{\hat{s}\hat{u}}{\hat{t}^2} - \frac{\hat{s}\hat{t}}{\hat{u}^2} \right)$\rule{0pt}{3.5ex}
	\end{tabular}
	\caption{Parton-Parton differential cross sections ($2\ \rightarrow\ 2$ QCD process), can beh calculated in pQCD evaluating the matrix element for each process involving quark, antiquark and gluon.}
	\label{table:partonic_cross_sections}
\end{table}

\subsection{Impact Parameter Dependence}

The simplest hypothesis for the multiple interaction simulation, it is to assume the same initial state for all hadron collisions without dependencies on the impact parameter. 

The more realistic scenario it is to include the possibility that each collision could be characterized by a different impact parameter $b$, where a small $b$ value correspond to a large overlap between the two hadrons this is related to the probability of parton-parton interaction to take place.

It is neccesary to make some assumption on the matter distribution inside the proton. A possibility is to assume a spherically simmetric distribution inside a hadron at rest $\rho(\mathbf{x})\,d^3x=\rho(r)\,d^3x$. A Gaussian ansatz is the most simple choice but it appears to lead to a narrow multiplicity distribution and too little of a pedestal effect. So the coiche is a double Gaussian:
\begin{equation}
	\rho(r)\propto \frac{1-\beta}{a_1^3}\exp\left\{-\frac{r^2}{a_1^2}}
\end{equation}

\end{document}