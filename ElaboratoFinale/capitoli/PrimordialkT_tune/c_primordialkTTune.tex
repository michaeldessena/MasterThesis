\chapter{Tune for the Primordial $k_T$}
\label{chap:primordialkTtune}

The primordial $k_T$ is another important parameter in the description of the proton-proton collisions with Monte Carlo simulations. The primordial $k_T$ description in \textsc{pythia8} was introduced in \secRef{sec:Beam Beam Remnants and primordial kT}.
\\
The main unresolved problem for the primordial $k_T$ tune is the unexpected high value required for the description of the observed $Z$ boson $p_T$ spectrum.
\\
In fact, the primordial $k_T$ found is origin in the Fermi Motion of the partons inside the hadrons. So when the parton undergoes to the hard scattering it can already have an initial no zero transverse momentum.
The value of the primordial $k_T$ can be estimate as reported in \eqRef{eq:PrimordialKT}, but in the case of the $Z$ boson $p_T$ spectrum this estimation is not sufficient the required value observed in order to reproduce the experimental data are in the order of $2\ \mathrm{GeV}$.
\\
%The introduction of the primordial $k_T$ is really important in the description of the $Z$ boson $p_T$ spectrum. At the LO calculation there is nothing that can recoil against the $Z$ boson so it can only be produced with zero $p_T$. With the introduction of the primordial $k_T$ one have that the already at the LO the $Z$ can have a non zero primordial $k_T$.
%
The primordial $k_T$ was tune in the Monash tune \cite{Monash} and the following tunes, as CP5, inherited the values of the parameters from there.
\\
So the introduction of a new tune for the primordial $k_T$ is needed. In fact the CP5 tune it is known that does not describe well the $Z$ spectrum in the low-$p_T$ region \cite{CPtunes}. This is shown in \figRef{fig:CP5_notdescribeZjet} for the production cross-section in $Z$ plus jets events and in the figure \figRef{fig:CP5_notdescribeZ} for $Z$ boson production in DY observations. As we can se the CP5 tune is very far from experimental data values in the low regions. 
\\
The distribution for $Z+$jets events as a function of the variable $p_T^{bal}=\big|p_T^Z-\sum_{\text{jets}}p_T^{j}\big|$ is not well described in the first bin. While the one as a function of the $p_T^Z$ miss the description in the region $p_T^Z \lesssim 40\ \mathrm{GeV}$. This two observable as described in \cite{CPtunes} are very sensitive to the parton shower process and UE.
\\
For the tune we focus on the distribution in \figRef{fig:CP5_notdescribeZ} distributions, also here we can see that CP5 tune don't describe very well the low regions for this two observables.  


\begin{figure}[!htb]
	\centering
	\noindent
	\begin{subfigure}{0.5\textwidth}
		\centering
		\includegraphics[width=0.985\textwidth]{{img/CPpaper_zjet1.png}}
	\end{subfigure}%
	\begin{subfigure}{0.5\textwidth}
		\centering
		\includegraphics[width=\textwidth]{{img/CPpaper_zjet2.png}}
	\end{subfigure}%
	\caption{Figure from CP tunes presentation paper \cite{CPtunes}. These two images show the $Z$ boson production cross-section in $Z$ plus jets (with at least one jet) at $\sqrt{s}=13\ \mathrm{TeV}$ as a function of the imbalance of the transverse momentum between the jet and the $Z$ boson (left) and of the $Z$ boson transverse momentum (right) \cite{CMS:2018mdf}. The first bin in the left distribution and the regione $p_T^Z \lesssim 40\ \mathrm{GeV}$ in the right one are not well described by CP5 tune (yellow line).}
	\label{fig:CP5_notdescribeZjet}
\end{figure}

\begin{figure}[!htb]
	\centering
	\noindent
	\begin{subfigure}{0.5\textwidth}
		\centering
		\includegraphics[width=\textwidth]{{img/rivet-plots-PrimordialkT_only_CP5default/CMS_2019_I1753680/d27-x01-y03.pdf}}
	\end{subfigure}%
	\begin{subfigure}{0.5\textwidth}
		\centering
		\includegraphics[width=\textwidth]{{img/rivet-plots-PrimordialkT_only_CP5default/CMS_2019_I1753680/d28-x01-y03.pdf}}
	\end{subfigure}
	\caption{The CP5 tune is not good in the description of the low region of the $Z$ boson production cross-section of the $Z$ boson transverse momentum (left) or of the $\phi_\eta^*$ angle. This distribution are from the CMS analysis \cite{ZpT_distributions} at the center-of-mass energy of $13\ \mathrm{TeV}$.}
	\label{fig:CP5_notdescribeZ}
\end{figure}

\section{Primordial $k_T$ and ISR effect on $p_T^Z$}

The parameters on which we focus in order to tune data and explain the $Z$ boson $p_T$ spectrum are:
\begin{itemize}
\item \texttt{BeamRemnants:primordialKThard} that set the width of the Gaussian distribution for the primordial $k_T$ sample;
\item \texttt{SpaceShower:pT0Ref} that set the threshold for the initial state radiation to take place.
\end{itemize}
Let's see how this parameter impact on the $Z$ boson production. If we consider only the LO diagram for the $Z$ production we can have only a production with a zero transverse momentum. So the $Z$ spectrum we expect at LO is a $\delta$-distribution function.  

\begin{figure}[!htb]
	\centering
	\includegraphics[width=0.8\textwidth]{{img/feynman_ZpT_2_blue.pdf}}
\end{figure}

With the addition of the primordial $k_T$ we get a Gaussian distribution which width is set by \texttt{BeamRemnants:primordialKThard}.
If we move to higher order diagrams the $Z$ boson can be produced with a non zero $p_T$ that is balanced by the various jets, then the primordial $k_T$ is added to the already non-zero $Z$ boson transverse momentum. 

A large contribution come also from the amount of ISR that is emitted from the incoming partons, in fact each split can give to the incoming parton a non-zero initial value of $p_T$ and so the $Z$ boson is created with a non-zero $p_T$ before the introduction of the primordial $k_T$. 

An higher amount of ISR, so a lower value for the threshold, can lead to a larger $p_T$ taken from the parton that generate the $Z$ boson with an higher $p_T$ and vice versa.

\section{Primordial $k_T$ tune}

The primordial $k_T$ was not investigated by CP5 so it is important to include that in the analysis in order to describe better the low $p_T$ region. 

The parameters variation ranges we use are the one reported in the \tableRef{table:primordialkT_variations} while others parameters are set to CP5 values.



\begin{table}[!htb]
\centering
\begin{tabular}{l | c }
Parameter Name & Value \\ 
\hline \hline
\\[-0.85em]
	\texttt{BeamRemnants:primordialKThard} & $[0.5 - 5.0]$\\[2pt]
	\texttt{SpaceShower:pT0Ref} & $[0.5 - 5.0]$\\
\end{tabular}
\caption{Variation ranges for the sampling used in the primordial $k_T$ tune.}
\label{table:primordialkT_variations}
\end{table} 

The number of samples is lower than the one used for the Underlying Event in fact the training set we use for the PerBin Model contains more or less $160$ MC runs and a little more for the Inverse Model, $\sim 200$.

\figRef{fig:CP5_notdescribeZ} shows the distributions used to performed the tune. The distributions  are from the \cite{ZpT_distributions} analysis at $\sqrt{s}=13\ \mathrm{TeV}$:
\begin{itemize}
	\item The $Z$ boson production cross-section in DY observation as a function of $p_T^Z$;
	\item The $Z$ boson production cross-section in DY observation as a function of $\phi_\eta^*$.
\end{itemize}

\noindent For the tune we use only the 


\medskip

\noindent The PerBin Model minimization results are displayed in \figRef{fig:result_PerBin_PrimordialkT_1}. The blue line indicates the value of the $\chi^2/DoF$ as a function of the parameter value. The best estimation is marked by a solid red line while the dashed red lines indicate the errors. Both the parameters are in a well defined minimum. 

\begin{figure}[!htb]
	\centering
	\noindent 
	\begin{subfigure}{0.48\textwidth}
	\centering
		\includegraphics[width=\textwidth]{{img/chi2_0.png}}
	\end{subfigure}%
	\begin{subfigure}{0.48\textwidth}
	\centering
		\includegraphics[width=\textwidth]{{img/chi2_1.png}}
	\end{subfigure}
	\caption{The output of the minimizer for the PerBin model in the primordial $k_T$ tune. The left panel shows the result of the minimization for the \texttt{SpaceShower:pT0Ref} parameter while the right one \texttt{BeamRemnants:primordialKThard}. The blue line is the $\chi^2/\mathrm{DoF}$ the best value is indicated by the solid red line while the errors are the ones indicated by the dashed red lines.}
	\label{fig:result_PerBin_PrimordialkT_1}
\end{figure}


The Inverse model distribution of prediction for the best configuration is reported in \figRef{fig:PrimordialkT_InverseModel_results} the best estimation for the parameter is indicated by the solid black line while the dashed black lines are the errors computed using the standard deviation. The hyperparameters space scanned to search for the best configuration is the same used above and reported in \tableRef{table:hyperpar_MinBias_2par}. The best hyperparameters we found are the one reported in the \tableRef{table:hyperpar_PrimkT}


\begin{figure}[!htb]
	\centering
	\includegraphics[width=0.95\textwidth]{{img/prediction_spread.png}}
	\caption{The prediction spread for the Inverse model in the primordial $k_T$ tune. The left histograms refers to the \texttt{BeamRemnants:primordialKThard} parameter while the right one to the \texttt{SpaceShower:pT0Ref}. The predictions are indicated by the solid black vertical lines and the errors by the dashed black lines. }
	\label{fig:PrimordialkT_InverseModel_results}
\end{figure}

\begin{table}[!htb]
	\centering
	\begin{tabular}{ l | c }
	Hyperparameter & Value\\[2pt]\hline\hline
	Number of hidden layer & 2 \\[2pt]
	Units layers & [15,\,17] \\[2pt]
	Activation function & sigmoid \\[2pt]
	Optimizer & rmsprop\\[2pt]
	Epochs & 2000\\[2pt]
	Batch size & 64\\[2pt]
	\end{tabular}
	\caption{Best hyperparameters model found for the primordial $k_T$ tune with Inverse model.}
	\label{table:hyperpar_PrimkT}
\end{table}


\noindent The PerBin Model gives to us very stable results the value and the errors we got from the tune are reported in \tableRef{table:Primordial_kT_results}.  
This are also compared to the ones obtained from the Inverse model. The value obtained from the two model are compatible each other. The default values for CP5 are also reported and they are the one inherited by Monash13 tune. As we can see the value we get are quite different to the one in CP5.

\begin{table}
	\centering
\begin{tabular}{l | c | c | c}
Parameter Name & PerBin & Inverse & Default CP5\\ 
\hline \hline
\\[-0.85em]
	\texttt{BeamRemnants:primordialKThard} & $ 2.5^{+0.2}_{-0.2} $ & $ 2.42\pm0.08 $ & $1.8$\\
	\texttt{SpaceShower:pT0Ref} & $ 1.6^{+0.5}_{-0.5} $ & $ 1.21\pm0.02  $ & $2.0$
\end{tabular}
\caption{Results obtained from the PerBin and the Inverse model in the tuning of the low region of $Z$ boson production spectra. they are compared to the default for CP5 (inherited from Monash tune.)}
\label{table:Primordial_kT_results}
\end{table}	


The overall results are displayed in \figRef{fig:result_primKT_FXFX}. Our tunes describe better the low region of the two spectrum. These low regions are the one we actually tune the higher $p_T$ region is only simulated using the value obtained from the tune of the first 5 bins in each distribution of  \figRef{fig:result_primKT_FXFX}. To simulate all the spectrum we have to use a mergin scheme as FxFx in order to avoid the double counting. 

 
%\begin{figure}[!htb]
%	\centering
%	\noindent
%	\begin{subfigure}{0.48\textwidth}
%	\centering
%	\includegraphics[width=\textwidth]{{img/rivet-plots-PrimordialkT_PerBin_vs_Inverse_vs_CP5_FxFx_weights/CMS_2019_I1753680/d27-x01-y03.pdf}}	
%	\end{subfigure}%
%	\begin{subfigure}{0.48\textwidth}
%	\centering
%	\includegraphics[width=\textwidth]{{img/rivet-plots-PrimordialkT_PerBin_vs_Inverse_vs_CP5_FxFx_weights/CMS_2019_I1753680/d28-x01-y03.pdf}}	
%	\end{subfigure}
%	\label{fig:result_primKT_FXFX}
%\end{figure}

\begin{figure}[!htb]
	\centering
	\noindent
	\begin{subfigure}{0.48\textwidth}
	\centering
	\includegraphics[width=\textwidth]{{img/rivet-plots-PrimordialkT_PerBin_vs_Inverse_vs_CP5_FxFx_noWeights/CMS_2019_I1753680/d27-x01-y03.pdf}}	
	\end{subfigure}%
	\begin{subfigure}{0.48\textwidth}
	\centering
	\includegraphics[width=\textwidth]{{img/rivet-plots-PrimordialkT_PerBin_vs_Inverse_vs_CP5_FxFx_noWeights/CMS_2019_I1753680/d28-x01-y03.pdf}}	
	\end{subfigure}
	\caption{The results we get from the tune of the primordial $k_T$. The left figure shows the $Z$ boson production cross-section as a function of the $p_T^Z$.  
The red line refers to the PerBin model tune, the blue line to the Inverse Model and the green line is the CP5 default tune. The black points are the experimental data. The colored vertical lines the statistical uncertainties.
	It is clear that our two tunes better describe the low region respect to the original CP5. }
	\label{fig:result_primKT_FXFX}
\end{figure}

\section{Primordial $k_T$ tune vs MPI}

As discussed above primordial $k_T$ is not the only source of non zero transverse momentum in a LO $Z$ boson production.
This is the reason why we are tuning also the Initial State Radiation. In fact the ISR can leads to a non zero initial $k_T$ a larger amount of ISR (low \texttt{SpaceShower:pT0Ref}) is related to more splitting occurring in the evolution of the incoming partons and before the hard scattering in which the $Z$ boson is produced. But this is not the only one process that can leads to a non-zero initial transverse momentum, there are also the MPI that can contribute to this initial transverse momentum.

\figRef{fig:result_primKT_MPI1} shows the effect of the MPI on the distributions we use for the tune. The blue line is obtained setting \texttt{PartonLevel:MPI=off} in \textsc{Phytia8} configuration.

\begin{figure}[!htb]
	\centering
	\noindent
	\begin{subfigure}{0.48\textwidth}
	\centering
	\includegraphics[width=\textwidth]{{img/rivet-plots-PrimordialkT_with_without_MPI_yellowBand/CMS_2019_I1753680/d27-x01-y03.pdf}}	
	\end{subfigure}%
	\begin{subfigure}{0.48\textwidth}
	\centering
	\includegraphics[width=\textwidth]{{img/rivet-plots-PrimordialkT_with_without_MPI_yellowBand/CMS_2019_I1753680/d28-x01-y03.pdf}}	
	\end{subfigure}
	\caption{The image shows that MPI have an effect on the two distribution. The blue line is the result without the MPI and the red line with MPI.}
	\label{fig:result_primKT_MPI1}
\end{figure}

The main parameters that controls the number of parton interactions in a single hadron-hadron collision is the \texttt{MultipartonInteractions:}\-\texttt{pT0Ref} parameter. It is clear that the first $2$ or $3$ bins of the left distribution of \figRef{fig:result_primKT_MPI1_pt0} are sensitive to the variation of this parameter. A future thing to do is to investigate the effects of these MPI in more detail and perform a more general tune including them in this description.


\begin{figure}[!htb]
	\centering
	\noindent
	\begin{subfigure}{0.48\textwidth}
	\centering
	\includegraphics[width=\textwidth]{{img/rivet-plots-primordialkT_vs_MPI_finale/CMS_2019_I1753680/d27-x01-y03.pdf}}	
	\end{subfigure}%
	\begin{subfigure}{0.48\textwidth}
	\centering
	\includegraphics[width=\textwidth]{{img/rivet-plots-primordialkT_vs_MPI_finale/CMS_2019_I1753680/d28-x01-y03.pdf}}	
	\end{subfigure}
	\caption{The image shows that MPI have an effect on the two distribution. The blue line is the result without the MPI and the red line with MPI.}
	\label{fig:result_primKT_MPI1_pt0}
\end{figure}


