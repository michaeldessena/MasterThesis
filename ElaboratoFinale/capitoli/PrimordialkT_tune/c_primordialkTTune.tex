\chapter{Tune for the Primordial $k_T$}
\label{chap:primordialkTtune}

The primordial $k_T$ is another important parameter in the description of the proton-proton collisions with Monte Carlo simulations. The primordial $k_T$ description in \textsc{pythia8} was introduced in \secRef{sec:Beam Beam Remnants and primordial kT}.
\\
The main unresolved problem for the primordial $k_T$ tune is the unexpected high value required for the description of the observed $Z$ boson $p_T$ spectrum.
\\
In fact, the primordial $k_T$ found is origin in the Fermi Motion of the partons inside the hadrons. So when the parton undergoes to the hard scattering it can already have an initial no zero transverse momentum.
The value of the primordial $k_T$ can be estimate as reported in \eqRef{eq:PrimordialKT}, but in the case of the $Z$ boson $p_T$ spectrum this estimation is not sufficient the required value observed in order to reproduce the experimental data are in the order of $2\ \mathrm{GeV}$.





