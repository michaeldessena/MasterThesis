Soft QCD studies require the use of Monte Carlo generators based on phenomenological models: these models introduce a lot of free parameters that have to be tuned with real data.
Due to the high computational cost of running a generator it is important to reduce the number of Monte Carlo runs required, this is done employing a parametrization-based approach where the real response of the generator is replaced by a surrogate one. The simplest approach is the use of a polynomial parametrization. 
In this work, an alternative approach based on machine learning techniques, implemented in the Python package \textsc{mcnntunes} by means of Feed-Forward Neural Networks, is used to tune parameters. 

\medskip
This thesis focuses on the analysis of proton-proton collisions obtained thanks to the Large Hadron Collider (LHC) at energy of $\sqrt{s}=13\ \mathrm{TeV}$ and detected by the CMS experiment. The charged particle tracks of the resulting, complex final state are detected by the CMS inner tracker. In addition to the activity from the main hard scattering, lots of other tracks are collected. This extra activity is called underlying event. The study of the  underlying event is important to describe the topology of a real proton-proton collision. The underlying event is studied employing Monte Carlo methods in the transverse regions defined from the products of the main hard scattering. These regions are less affected by the hard interaction effects.
\\
The contribution to the parton momentum of the so-called primordial $k_T$, the transverse motion originating from the Fermi motion of partons inside of hadrons, is also studied within this thesis work. The observables sensitive to this contribution are the $p_T$ and other characteristic distributions of $Z$-boson production.
\medskip

Firstly, the basic theoretical concepts used in the Monte Carlo simulations are introduced.
The aim of our work was to validate \textsc{mcnntunes} as a tool for the tuning of Monte Carlo generators in High Energy Physics simulations.
To perform this validation, we try to reproduce an already existing tune for the description of the Underlying Event in hadron-hadron collisions using data collected in Minimum Bias events.
So, in the first part the parameters of the \textsc{pythia}8 event generator related to the Multi Parton Interactions and Color Reconnection  have been tuned. The tune was performed employing data from the CDF and CMS experiments at different center-of-mass energies. The validation of the \textsc{mcnntunes} tool is performed in two steps: a first test with a limited number of free parameters in order to test all the functionalities and check the different operation mode in a simpler case than the actual tune. Once the tool as been tested, the actual tune is performed. 

\medskip

In the second part of the thesis, the \textsc{mcnntunes} tool is employed to the tune of the Primordial $k_T$ and Initial State Radiation parameters using data collected in $Z$ boson-production events by the CMS experiment at the center-of-mass energy of $13\ \mathrm{TeV}$. This is the first test of investigating the feasibility of such a tuning in the CMS collaboration.


%the parameters that describe Multi Parton Interactions and Color Reconnection, in Pythia event generator framework, are tuned: this is done employing observables sensible to  underlying events in minimum bias observation, as the charged particle density and the charged particle scalar momentum sum in the two transverse regions defined respectively to the direction of the leading object arising from the hard scattering of two partons. 
%The aim of this part is to reproduce the result obtained by CMS collaboration that employed the polynomial parametrization. 
%
%Lastly, the primordial $k_T$ is tuned in Z production observables as: Z production cross section at low $p_T$ with or without an associated jet in the event. It is investigate the the idea that the contribution of this primordial $k_T$ is decoupled from the contribution of Multi Parton Interaction in this low $p_T$ region.