\chapter{Tune procedure, CP5 Tune and \textsc{mcnntunes}}
\label{chap:TuneprocedureCP5TuneandMCNNTUNES}

The study of the underlying event, or more in general of softQCD processes require the use of Monte Carlo generator based on phenomenological model all these models introduce lot of free parameters that must be tuned on experimental data in order to obtain meaningful results. The procedure of estimate the best parameters values is called \textit{tune}. 

The tune procedure can be really computational expensive,  can require to run the generator a very large amount of times. And usually these MC generator are really expansive in computation time required for a single job. 

To tune some parameters the number of jobs you have to run increase with the number of parameter you want to tune. different approaches have been used in the past to tune these MC generator:
\begin{enumerate}[label=\arabic*)]
	\item \textbf{By eye approach}:
	\item \textbf{Brute force approach}:
	\item \textbf{Parametrization-based approach}:
\end{enumerate} 