
\chapter{Observable to Study the Underlying Event}
\label{chap:ObservabletoStudytheUnderlyingEvent}

The underlying events UE are all the processes not associated with the primary hard scatter in an hadron-hadron collision.
\\
All the process described in the previous section: ISR, FSR, MPI, and BBR and their interactions with color exchanges, contribute to the Underlying Event (UE) in the proton-proton collision.
\\
As an example, in a proton-proton collision with a $Z$ boson production, that then can decay in two leptons ($l^+$ and $l^-$) and in the end these two leptons observed by the experiment, we have that the hard scattering is represented from the scatters between the two partons that generate the $Z$ boson. While, the so called UE is represented from all the extra scatterings, the various radiations and in general all the activity not associated with this primary hard scattering.  

\medskip

It is important to underline the fact that most of the observables to study the UE are sensible only to the sum of these contributions and not to the single ones. So, a good description of all these processes and their interplay is really important in order to study the complexes finals states originating from this scatters that contribute to the observables.
\\
In this chapter these observables sensible to the UE are introduced and described with more detail.

\section{Minimum Bias Measurements and Underlying Event topology}

A Minimum Bias (MB) measurement is a collection of inelastic events with a loose event selection. Which means the events are collected requiring the minimal interaction with the detector (the smallest possible bias). The most of the interactions in MB observation are soft, $p_T\lesssim2\ \mathrm{GeV}$, but the study of the UE requires at least one hard scattering ($p_T\gtrsim2\ \mathrm{GeV}$) presence; in fact the UE is given by the underlying activity to a primary hard scatter.
\\
To study the UE the topological structure of an hadron-hadron collision is used. The analysis is performed on an \textit{event-by-event} basis. In the analysis the direction of the leading object is used to define regions in the $\eta-\phi$ space. Where $\eta$ is the pseudorapidity defined as $\eta=-\log{\tan\left(\frac{\theta}{2}\right)}$, while $\phi$ is the azimuthal angle in the $x-y$ plane. 
The last one is defined from the direction of the leading object as $\Delta\phi=\phi-\phi_{\text{leading}}$ .
%
\begin{figure}[!htb]
	\centering
	\includegraphics[width=0.8\textwidth]{{img/Regions.pdf}}
	\caption{This figure shows the four regions for the description of the UE on a event-by-event analysis. The angular values, defining the four regions, are shown in the table. The regions are defined starting from the leading object direction. The toward and away regions contain most of the contribution from the hard scattering (e.g. in a $t\overline{t}$ production event the two quark $t$ are located in these regions); while, the transverse region are the ones in between the two other regions, these are the most important for the study of the UE.}
	\label{fig:Regions}
\end{figure}
%
\\
The regions classification is shown in \figRef{fig:Regions}, we have:
\begin{itemize}
	\item \textbf{Toward region}: the region that contains the leading object, this region contains the most of the particles produced by the hard interaction.
	\item \textbf{Away region}: this region contains the objects that recoil against the leading object, also this region contains mostly the particles produced by the hard interaction.
	\item \textbf{Transverse regions}: the two transverse regions are the most sensitive to UE.
\end{itemize}
The toward and away regions are the ones with the major contribution from the primary hard scattering, in fact in a dijets event the leading jet is expected in the toward region while the next-to-leading jet in the away region.  
\\
The two transverse regions are called:
\begin{itemize}
	\item[--] \textbf{TransMAX}: This is the transverse region that contains the \textit{maximum} number of charged particles, or scalar $p_T$ sum of charged particles. This regions includes both MPI and hard-process contamination.
As an example, in events with 3-jets production these region can contains the extra jet.	
	\item[--] \textbf{TransMIN}: is the transverse region that contains the \textit{minimum} number of charged particles, or scalar-$p_T$ sum of charged particles. This region is the most  sensitive to MPI effects.
\end{itemize}

\noindent The leading object definition depend on the type of event under observation and defer from one analysis to another. 
The charged-particle with largest $p_T$ \cite{CMS-PAS-FSQ-15-007}, the dilepton system in Drell-Yan observation \cite{CMS:2012oqb, CMS:2017ngy} or $t\overline{t}$ events \cite{CMS:2018mdd} can all be used as leading object in the analyses for the UE event.
\\
So we are interest in studying the transMAX and transMIN regions in particular the observable sensitive to UE in these regions. The main observables studied in these regions are: the charged-particle density and the charged-particle scalar-$p_T$ sum density in the $\eta-\phi$ space.

In an hadron-hadron collision with two jets productions it is observed that not only the toward and away regions activity increases with the hardness of the collision ($p_T^{max}$) but also the transverse regions activity increases with it. This is shown in \figRef{fig:UEpredictions_in_regions.png}, where the different color lines refer to different scales for the collision. It is observed that the away region ($60\leq |\Delta\phi|\leq 120$) at increasing energy for the collision become broader this is related to the increasing quantity of FSR.
\\
This increment in the transverse regions activity cannot be explained only by the increase of the FSR. To explain this rise one need to attribute activity in the transverse regions to MPI: the number of extra scatters increases with energy. This rise is related to the density functions for the partons inside the hadrons that increse when probed at higher energies and so the partons become denser packed in hadrons. In this way the extra scatters probability is larger to higher energies. 

\begin{figure}[!htb]
	\centering
	\includegraphics[width=0.8\textwidth]{{img/UEpredictions_in_regions.png}}
	\caption{A comparison between three different scales for the interaction, in \textsc{pythia}. The multiplicity of charged particles (left) and the scalar-$p_T$ sum of charged particles (right) are simulated. The activity in the transverse regions increase due two effects: the FSR this is related to the broadening of the away region so some events from the shower end up in the transverse region (yellow band) but this alone can't explain the increment so is required the introduction of the MPI in the description. Figure from chapter 5 of \cite{Bechtel:2009zza}.}
	\label{fig:UEpredictions_in_regions.png}
\end{figure}

\noindent The evolution of these two quantities in transMAX region as a function of the transverse-momentum of the leading object (measurement of the energy scale for the collision) is shown in more detail in \figRef{fig:CPtransmax_evolution_with_energy}. The two distributions show a rapid raise at low leading-object $p_T$ than a very low rise starts.
%
\begin{figure}[!htb]
	\centering
	\includegraphics[width=0.9\textwidth]{{img/CPdistributions.png}}
	\caption{The evolution of the charged-particles density (left) and of the scalar-$p_T$ sum density of charged particles (right) as a function of the energy scale for the scattering (leading object $p_T$), the two distribution increase quite rapidly in the first bins than they saturate ($\approx 6-7\ \mathrm{GeV}$) the scalar-$p_T$ sum density increase a little bit more, for $p_T>$ but very slowly. The black dots are the experimental point and are compared to prediction with CMS \textsc{pythia} tunes: CP3, CP4 and CP5; they are described with more details in the next chapter. Figure from \cite{CPtunes}.}
	\label{fig:CPtransmax_evolution_with_energy}
\end{figure}

Now we want to look for the sensitivity of these observables to some \textsc{Pythia8} parameters. In figure \figRef{fig:CP5onlyPT0} is shown the effect of the MPI on booth the distributions shown before separately for TransMAX and TransMIN regions. 
%The amount of the MPI is related to the value of $p_{T0}$: an high value of $p_{T0}$ is related to less MPI (blue line) and an higher value to a major contribution from the the MPI (red line).
The figure shows the case with MPI (red line) and the case when the MPI are switched off (blue line). As expected the MPI are needed to explain the activity in the transverse regions. When the MPI are switched-on the number of particle produced is higher than the case when they are off. From this figure is clear that MPI are the main contribution in the activity of these TransMAX and TransMIN regions. It's obviously that the discrepancy increases as the hard scale of the interaction increases, in fact in low $p_T$ regions the $p_T$ evolution start at a lower value so the phase-space, for the extra collisions to occur, is smaller.


\begin{figure}[!htb]
	\centering
	\includegraphics[width=0.9\textwidth]{{img/CP5_with_without_MPI.pdf}}
	\caption{This image shows the effect of the MPI in the transMAX and TransMin regions. The contribution of the parton shower alone can't explain the contributions of the underlying event in these two regions (blue line) the introduction of the contribution from the MPI is necessary (red line). The two simulations are compared to the data from \cite{CMS-PAS-FSQ-15-007}.}
	\label{fig:CP5onlyPT0}
\end{figure}

Another important observation that was also pointed out in the \chapRef{sec:BasicConcepts} is that the amount of activity in the two transverse regions is also dependent on the center-of-mass energy. This evolution is shown in \figRef{fig:CPECMdep} for two different energy. The amount of MPI increase with $\sqrt{s}$ that was expected from the evolution of the PDF with the energy of the collision: the hadrons become denser-packed when probed to higher energies. In \figRef{fig:CPECMdep} the charged particle density and the scalar sum of the transverse momentum in the transverse regions data are compared for two different center-of-mass energies. The data are also compared to existing different CMS tuning.

\begin{figure}[!htb]
	\centering
	\includegraphics[width=0.9\textwidth]{{img/CPECMdep.pdf}}
	\caption{The charged particle density in the transMAX (upper left) and transMIN (lower left) regions and of the charged particle $p_T$-sum in the transMAX (upper right) and the transMIN (lower right) regions evolution as function of the center-of-mass energy is shown. The red ones are the data for $\sqrt{s}=2.76$ and the blue ones for $\sqrt{s}=13$ and these are compare to different CMS tunes. Figure from \cite{CMS-PAS-FSQ-15-007}}
	\label{fig:CPECMdep}
\end{figure}

So these observables are the main ones we are going to use in our tune and used in previous tunes for the UE. The data from the activity in the TransMAX and TransMIN  regions have been collected by  a large number of experiment at different center-of-mass energies.

\subsection{Observables in Z production processes}

In the last part of this work we are going to tune the primordial $k_T$ another important parameter in the simulation of hadron-hadron collisions. An unresolved question is that the primordial $k_T$ value required for the description of the observed cross section for the Z production is very large respect to the expected value derived from the typical size of an hadron in \eqRef{eq:PrimordialKT}. This very high value is not understood by first principles so it is required a tune of this parameter in the Monte Carlo generator in order to describe the data in a good way. 
\\
The observables to study the primordial $k_T$ are the differential cross section for Z production in proton-proton collision in Drell-Yan observation as a function of the transverse momentum of the Z boson (\figRef{fig:PrimordialkT_distributionUsed_noFxFx} left) and as a function of the $\phi^*_\eta$ angle (\figRef{fig:PrimordialkT_distributionUsed_noFxFx} right). Where the angle $\phi^*_\eta$ is defined as:
\begin{equation}
	\phi_\eta^*=\tan\left(\frac{\pi-\Delta\phi}{2}\right)\,\sin(\theta_\eta^*)\quad,
	\label{eq:def_phistar}
\end{equation}
where $\Delta\phi$ is the azimuthal angle between the two leptons. The angle $\theta_\eta^*$ instead is the angle measured respect to the beam direction in the rest frame of the dilepton system (frame with leptons emitted back-to-back), so is defined as: 
\begin{equation}
	\cos(\theta_\eta^*)=\tanh\left(\frac{\eta^--\eta^+}{2}	\right)\quad,
\end{equation} 
with $\eta^+$ and $\eta^-$ the pseudorapidities of the two leptons \cite{phistareta}.

\begin{figure}[!htb]
	\centering
	\noindent
	\begin{subfigure}{0.5\textwidth}
	\centering
	\includegraphics[width=0.97\textwidth]{{img/Angle1.pdf}}
	\end{subfigure}%
	\begin{subfigure}{0.5\textwidth}
	\centering
	\includegraphics[width=0.97\textwidth]{{img/Angle2.pdf}}
	\end{subfigure}
	\caption{A schematic representation of a Drell-Yan observation with a description of the angles used in the definition of $\phi_\eta^*$ in \eqRef{eq:def_phistar}. }
	\label{fig:angles}
\end{figure}

To full simulate the Z production spectrum we have to merge matrix element calculation and space shower to do this we used the FxFx merging scheme that have been introduced in \chapRef{chap:Hadron-HadronScattering}. 
As shown in \figRef{fig:PrimordialkT_distributionUsed_noFxFx} the description with out FxFx is good only in the first bins. The strategy we follow for the tuning of the primordial kT is tu tune the Primordial $k_T$ in this low region. From these two distribution we take only the first 5 bins each.
\\
Than we are going to run the FxFx only for the comparison once the parameters are tuned. This is going to be discussed in \chapRef{chap:primordialkTtune} where we perform the primordial $k_T$ tune.

\begin{figure}[!htb]
	\centering
	\includegraphics[width=0.9\textwidth]{{img/PrimordialkT_distributionUsed_noFxFx.pdf}}
	\caption{Here are shown the distribution, from the \citep{ZpT_distributions} CMS analysis at $\sqrt{s}=13\ \mathrm{TeV}$, that we are going to use for the tune of the primordial $k_T$. The left one is the $Z$ boson production cross section as a function of the transverse momentum of the $Z$ boson. On the right the differential cross section for the $Z$ boson production as a function of the $\phi_\eta^*$ angle. The red line is the the \textsc{pythia8} description before the tune of the primordial $k_T$ and without the FxFx merging scheme so only the first bins have to be taken into account.}
	\label{fig:PrimordialkT_distributionUsed_noFxFx}
\end{figure}

\bigskip

Next chapter discusses the general approaches to the tune and why we use machine learning and \textsc{mcnntunes}. \textsc{mcnntunes} is also introduced together with the already existing CP* tunes in particular the CP5 tune.


