\chapter{Multiple Parton Interactions}



The fact that the hadrons are viewed as "bunches" of partons it is likely that in an hadron-hadron different couples of partons can undergo to a scattering. This phenomenon is know as Multiple Parton Interactions (MPI) and is related to the composite nature of the incoming hadrons. 

So, at some level these MPI have to exist, and are important in the description of the event: They can change the color topology of the colliding system as a whole,  

In this scenario it is important to have a good understanding of the phenomenon. The aim of this section is to describe the basic concepts that are use to simulate MPI, for example in \texttt{Pythia} Monte Carlo event generator.

\section{Basic Concepts}

The main hypothesis of the multiple interaction models is that the QCD factorization is true not only for the hard process but also for the other scatters.

So, we can write:

\begin{equation}
	\frac{d\sigma_{\text{int}}}{dp_\perp}=\displaystyle\sum_{i,j,k,l}\displaystyle\int dx_1 \displaystyle\int dx_2 \displaystyle\int d\hat{t}\, f_i(x_1,Q^2)f_j(x_2,Q^2)\frac{d\hat{\sigma}_{ij\,\rightarrow\,kl}}{d\hat{t}}\delta\left( p_\perp^2-\frac{\hat{t}\hat{u}}{\hat{s}} \right)
	\label{eq:sigma_int1}
\end{equation}


This is the differential cross section for QCD hard $2\ \rightarrow 2$ processes, this processes are the one reported in \tableRef{table:partonic_cross_sections}. 

In \eqRef{eq:sigma_int1} we use the Mandelstam variables associated to the partonic system:

\begin{align}
	&\hat{s}=(p_1+p_2)^2=(p_3+p_4)^2=x_1x_2s\\
	&\hat{t}=(p_1-p_3)^2=(p_2-p_4)^2\\
	&\hat{u}=(p_1-p_4)^2=(p_2-p_3)^2
\end{align} 
wherer $p_1$, $p_2$ are the four-momenta of the incoming partons and $p_3$, $p_4$ the four-momenta of the outgoing partons. 
Note that in \eqRef{eq:sigma_int1} the jet cross section is twice as large $\sigma_{\ \text{jet}}=2\sigma_{\text{int}}$, because at first approximation each interaction give rise to two jets.

We assume also that the "hardness" of processes is defined by the $p_\perp$ scale of the interaction ($Q^2=p_\perp^2$).

\begin{table}[!ht]
	\centering
	\begin{tabular}{l | c}\rule{0pt}{3ex} 
	Process & Partonic cross section \rule{0pt}{3.5ex}\\   \hline\hline 
	$q\,q'\ \rightarrow\ q\,q'$ & $\frac{4}{9}\frac{\hat{s}^2+\hat{u}^2}{\hat{t}^2}$\rule{0pt}{3.5ex}\\
	$q\,q\ \rightarrow\ q\,q$ & $\frac{4}{9}\left(\frac{\hat{s}^2+\hat{u}^2}{\hat{t}^2}+\frac{\hat{s}^2+\hat{t}^2}{\hat{u}^2}\right) -\frac{8}{27}\frac{\hat{s}^2}{\hat{u}\hat{t}}$\rule{0pt}{3.5ex}\\
	$q\,\overline{q}\ \rightarrow\ q'\,\overline{q}'$ & $\frac{4}{9}\frac{\hat{s}^2+\hat{u}^2}{\hat{t}^2}$\rule{0pt}{3.5ex}\\
	$q\,\overline{q}\ \rightarrow\ q\,\overline{q}$ & $\frac{4}{9}\left(\frac{\hat{s}^2+\hat{u}^2}{\hat{t}^2}+\frac{\hat{t}^2+\hat{u}^2}{\hat{s}^2}\right) -\frac{8}{27}\frac{\hat{u}^2}{\hat{s}\hat{t}}$\rule{0pt}{3.5ex}\\
	$q\,\overline{q}\ \rightarrow\ g\,g$ & $\frac{32}{27}\frac{\hat{t}^2+\hat{u}^2}{\hat{t}\hat{u}}-\frac{8}{3}\frac{\hat{t}^2+\hat{u}^2}{\hat{s}^2}$\rule{0pt}{3.5ex}\\
	$g\,g\ \rightarrow\ q\,\overline{q}$ & $\frac{1}{6}\frac{\hat{t}^2+\hat{u}^2}{\hat{t}\hat{u}}-\frac{8}{3}\frac{\hat{t}^2+\hat{u}^2}{\hat{s}^2}$\rule{0pt}{3.5ex}\\
	$g\,q\ \rightarrow\ g\,q$ & $-\frac{4}{9}\frac{\hat{s}^2+\hat{u}^2}{\hat{s}\hat{u}}-\frac{\hat{u}^2+\hat{s}^2}{\hat{t}^2}$\rule{0pt}{3.5ex}\\
	$g\,g\ \rightarrow\ g\,g$ & $\frac{9}{2}\left( 3-\frac{\hat{t}\hat{u}}{\hat{s}^2} -\frac{\hat{s}\hat{u}}{\hat{t}^2} - \frac{\hat{s}\hat{t}}{\hat{u}^2} \right)$\rule{0pt}{3.5ex}
	\end{tabular}
	\caption{Parton-Parton differential cross sections ($2\ \rightarrow\ 2$ QCD process), can beh calculated in pQCD evaluating the matrix element for each process involving quark, antiquark and gluon.}
	\label{table:partonic_cross_sections}
\end{table}

As you can see from the formulae in \tableRef{table:partonic_cross_sections}
at small scattering angles: for $t\,\rightarrow\,0$,  the t-channel gluon exchange processes $qq'\,\rightarrow\,qq'$, $qg\,\rightarrow\,qg$ and $gg\,\rightarrow\,gg$ dominate the full matrix element. For scattering that are soft relative to $\hat{s}$, $|\hat{t}|\ll \hat{s}$, we can approximate $|\hat{t}|$ as:
\begin{equation}
	p_T^2=\frac{\hat{t}\hat{u}}{\hat{s}} = \frac{\hat{t}(-\hat{s}-\hat{t})}{\hat{s}} \approx |\hat{t}|
\end{equation}
In this limit the only difference between quark and gluon are the color factors:
\begin{equation}
	\hat{\sigma}_{gg}:\hat{\sigma}_{qg}:\hat{\sigma}_{qq}=\frac{9}{4}:1:\frac{4}{9}
\end{equation}

So the \eqRef{eq:sigma_int1} can be rewritten as:
\begin{equation}
	\frac{d\sigma_{int}}{dp_T^2}\approx\int\int\frac{dx_1}{x_1}\,\frac{dx_2}{x_2}\,F(x_1,p_T^2)\,F(x_2,p_T^2)\frac{d\hat{\sigma}_{2\,\rightarrow\,2}}{dp_T^2}
	\label{eq:sigma_int2}
\end{equation}
whit:
	\begin{gather}
		\frac{d\hat{\sigma}_{2\,\rightarrow\,2}}{dp_T^2} = \frac{8\pi\alpha_s^2(p_T^2)}{9p_T^4}\\
		F(x,Q^2)=\displaystyle\sum_q\left( x\,q(x,Q^2) + x\,\overline{q}(x,Q^2) \right) + \frac{9}{4}\,x\,g(x,Q^2)
	\end{gather}

So we can integrate the \eqRef{eq:sigma_int2}:
\begin{equation}
	\sigma_{int}(p_{T\,min})= \displaystyle\int_{ p_{T\,min}^2}^{(\sqrt{s}/2)^2}\frac{d\hat{\sigma}_{2\,\rightarrow\,2}}{dp_T^2}\,dp_T^2\propto \frac{1}{p_{T\,min}^2}\  \xrightarrow{\ \ p_{T\,min}\,\rightarrow\, 0 \ }\ \ \infty
	\label{eq:interaction_crossSection}
\end{equation}

\begin{figure}[!ht]
	\centering
	\includegraphics[width=12cm]{{img/CrossSection_Tot.png}}
	\caption{ADD caption}
	\label{fig:CrossSection_Tot}
\end{figure}

The total cross section is divergent in the limit $p_T\,\rightarrow\,0$, this divergence is shown in \figRef{fig:CrossSection_Tot}. Due to this divergence the total interaction cross section at some $p_{T}$ scale can exceed the total proton-proton cross section.

\bigskip
-
To understand this paradox should be noted that the interaction cross section described in \eqRef{eq:interaction_crossSection} is related to the interaction probability between two partons and counts the number of interactions, while the total proton-proton cross section $\sigma_{pp}$ counts the number of events. For example, an event (a proton-proton collision) in which two partons interact counts once in the total cross section and twice in the interaction cross section.

So the ratio between this two quantities is perfectly allowed to be larger than unity:
\begin{equation}
	\frac{\sigma_{int}(p_{T\,min})}{\sigma_{tot}}=\langle n \rangle (p_T\,min)
\end{equation}
 
Secondly, we have to consider the screening effect: in fact the incoming hadrons are color singlet objects. Therefore , when the $p_T$ of an exchanged gluon is small, and so the associated wavelength large, this gluon can no longer resolve the color charges and the effective coupling is decreased.

This cutoff is associated with color screening distance i.e. the average size of the region within the  compensation of color charge occurs.
This cutoff is then introduced in the factor:
\begin{equation}
	\frac{\alpha_s^2\,(p_{T0}^2+p_{T}^2)}{\alpha_s^2\,(p_T^2)}\,\frac{p_T^4}{(p_{T0}^2+p_T^2)^2}
\end{equation}
This factor contains the phenomenological regularization of the divergence , with the factor $p_{T0}$ that have to be tuned to data. 

Now the interaction cross section i

This parameter $p_{T0}$ do not have to be energy-independent. But the energy is related with the sensibility of our probe, higher energy is related with the capacity of probing PDF to lower x values, this low x region the number of partons rapidly increases. So, the partons are closer packed is this regions and as a consequence the color screening distance $d$ decrease.

The number of partons is related to $x$ with a power low so it is likely to have a dependence of the same form for $p_{T0}$ respect to the CM energy.
\begin{equation}
	p_{T0}(\sqrt{s})=p_{T0}^{ref} \,\left( \frac{\sqrt{s}}{E_{CM}^{ref}} \right)^{E_{CM}^{pow}}
\end{equation}
	
\section{\texttt{Pythia8} Monte Carlo events generator}	

\texttt{Pythia}8 is a standard tool for the simulation of events in high energy collisions. \texttt{Pythia} contains the evolution from a few-body system to a complex multiparticle final state.

\subsection{Parton Shower}

 In \texttt{Pythia}8 all the contributions from Initial State Radiation (ISR), Final State Radiation (FSR) and Multi Parton Interactions (MPI) are interleaved into a single common sequence of decreasing $p_T$. 


The shower evolution is based on the standard (LO) DGLAP splitting kernels P(z)
\begin{align}
P_{q\,\rightarrow\,qg}(z) & = C_F\frac{1+z^2}{1-z} \\
P_{g\,\rightarrow\,gg}(z) & = C_A\frac{(1-z(1-z))^2}{z(1-z)} \\
P_{q\,\rightarrow\,q\overline{q}}(z) & = T_R(z^2+(1-z)^2)
\end{align} 
where $C_F=\frac{4}{3}$, $C_A=N_C=3$ and $T_R=\frac{1}{2}$ multiplied by $N_f$ if summing over all contributing quark flavours.

Both ISR and FSR algorithms are based on these splitting kernels.

The probabilities of emitting radiation as one moves in the common downward evolution are:
\begin{align}
	\frac{d\mathcal{P}_{FSR}}{dp_T^2} &= \frac{1}{p_T^2}\int dz\,\frac{\alpha_s}{2\pi}P(z)\\
	\frac{d\mathcal{P}_{ISR}}{dp_T^2} &= \frac{1}{p_T^2}\int dz\,\frac{\alpha_s}{2\pi}P(z)\,\frac{f'(x/z,p_T^2)}{z\,f(x,p_T^2)}
\end{align}

\subsection{Multiple Parton Interactions in \texttt{Pythia}}

The MPI, as said before, are also generated in a decreasing $p_T$ sequence. So the hardest MPI is generated first. Than we can write the probability for an interaction, $i$, to occur at a scale $p_T$ using a Sudakov-type expression:
\begin{equation}
	\frac{d\mathcal{P}_{MPI}}{dp_T}=\frac{1}{\sigma_{ND}}\frac{d\sigma_{2\,\rightarrow\,2}}{dp_T}\ \exp\left( -\displaystyle\int_{p_T}^{p_T\,i-1} \frac{1}{\sigma_{ND}}\frac{d\sigma_{2\,\rightarrow\,2}}{dp_T'}\,dp_T' \right)
\end{equation}

\subsection{Momentum and flavour conservation}

One problem is to achieve momentum conservation, so we need to take into account the modification in the PDF by the $i-1$ interaction. To do that in \texttt{Pythia} the PDF are rescaled to the remaining available $x$ range, adjusting their normalization.

We need to take into account the momentum fraction $x_i$ removed from the hadron remnant by the $i-th$ interaction. This is done evaluating PDF not at $x_i$ but at a rescaled value
\begin{equation}
	x_i'=\frac{x_i}{X} \qquad \ \text{with: }\ \ X=1-\sum_{j=1}^{i-1}x_j
\end{equation}

So using these quantities, we can rewrite our PDFs as:
\begin{equation}
	f_i(x,Q^2)\ \longrightarrow\ \frac{1}{X}f_0\left(\frac{x}{X}\right)
\end{equation}
where $f_0$ the original one-parton inclusive PDFs.

Now, requiring also the flavour conservation and taking into account the number of valence and/or sea quarks involved in the preceding MPI. We have now the full forms of the PDFs used for the $i-th$ MPI:
\begin{align}
f_i(x,Q^2) =&  \frac{N_{fv}}{N_{fv0}}\frac{1}{X} f_{v0}\left( \frac{x}{X},Q^2 \right) + \frac{a}{X}f_{s0}\left( \frac{x}{X},Q^2 \right)+\displaystyle\sum_j \frac{1}{X} f_{c_j0}\left( \frac{x}{X},Q^2 \right) \\
g_i(x,Q^2) =& \frac{a}{X}g_0\left( \frac{x}{X},Q^2 \right) 
\end{align}
where: 
\begin{itemize}
	\item $f_i(x,Q^2)$ $(g_i(x,Q^2))$ are the squeezed PDFs for quarks (gluons)
	\item $N_{fv}$ the number of remaining valence quarks of the given flavour
	\item $N_{fv0}$ the number of original valence quarks of the given flavour (for the proton we have $N_u=2$, $N_d=1$)
	\item $f_{s0}$ the sea-quark PDF
	\item $f_{cj}$ the companion PDF, this arise from the splitting $g\,\rightarrow\,q\overline{q}$ and a quark $j$ is kicked out.
\end{itemize}
The factor $a$ is defined to satisfy the total momentum sum rule.

\subsection{Impact Parameter Dependence}

The simplest hypothesis for the multiple interaction simulation, it is to assume the same initial state for all hadron collisions without dependencies on the impact parameter. 

The more realistic scenario it is to include the possibility that each collision could be characterized by a different impact parameter $b$, where a small $b$ value correspond to a large overlap between the two hadrons this is related to the probability of parton-parton interaction to take place.

It is neccesary to make some assumption on the matter distribution inside the proton. A possibility is to assume a spherically simmetric distribution inside a hadron at rest $\rho(\mathbf{x})\,d^3x=\rho(r)\,d^3x$. A Gaussian ansatz is the most simple choice but it appears to lead to a narrow multiplicity distribution and too little of a pedestal effect. So the coiche is a double Gaussian:
\begin{equation}
	\rho(r) \propto \frac{1-\beta}{a_1^3}\exp\left\{-\frac{r^2}{a_1^2}\right\}+\frac{\beta}{a_2^3}\exp\left\{ -\frac{r^2}{a_2^2} \right\}
\end{equation}
where a fraction $\beta$ of matter is contained in a radius $a_2$, and the rest in a larger radius $a_1$.

\bigskip

Now for a given matter distribution $\rho(r)$,  the time-integrated overlap function of the incoming hadrons during the collision is given by:
\begin{equation}
	\mathcal{O}(b)=\displaystyle\int dt \displaystyle\int d^3x\,\rho(x,y,z)\,\rho(x+b,y,z+t)
\end{equation}
this is useful to quantify the effect of overlapping protons.
The larger is $\mathcal{O}(b)$ the more probable are parton-parton scatters between the incoming protons. 

So, these assumption change the so-far Poissonian nature of the framework
\begin{equation}
\mathcal{P}(\widetilde{n})=
	\left\langle \widetilde{n}\right\rangle ^{\widetilde{n}}\ \frac{e^{-\langle\widetilde{n}\rangle}}{\widetilde{n}!}
\end{equation}
Now the request of at least one interactions and the introduction of an impact parameter $b$ modify the probability distribution of interactions number from a Poissonian to a narrower one.

\subsection{Parton rescattering}

It is not necessary that the partons undergoing to a MPI are a different partons couple from the one scattered before. As shown in \figRef{fig:Rescattering} MPI can also arise when a parton scatters more than once against partons from the other beam, this is call \textit{parton rescattering}. 

\begin{figure}[!ht]
	\centering
	\includegraphics[width=12cm]{{img/Rescattering.png}}
	\caption{Add caption}
	\label{fig:Rescattering}
\end{figure}

We can have 3 type of MPI:
\begin{enumerate}
	\item No ones of the partons that enter in the second scattering undergoes to another scatter before
	\item Only one of the two partons have alredy been scattered
	\item Both the partons have alredy been scattered before.
\end{enumerate}
The second and the third are the rescatters the overall influence of rescatters in proton-proton interactions was estimated to be small respect to the first case with distinct $2\,\rightarrow\,2$  scatters. 

The simulation of parton rescatters start from the evaluation of the parton density as:
\begin{equation}
	f(x,Q^2)\ \  \longrightarrow\! \underbrace{\phantom{\Bigg(} f_{\text{rescaled}}(x,Q^2)}_{\text{hadron remnant}}+\underbrace{\displaystyle\sum_n \,\delta(x-x_n)}_{\text{scattered parton(s)}}
\end{equation} 

The $\delta$ take into account the scattered partons that have a fixed momentum fraction $x_n$. While the hadron remnant is still described by a continuous momentum density, rescaled to achieve momentum conservation:
\begin{equation}	\displaystyle\int_0^1 \bigg( f_{\text{rescaled}}(x,Q^2) +\displaystyle\sum_n \,\delta(x-x_n) \bigg)dx=1
\end{equation}

\subsection{Interleaving of Multiple Interaction and Parton Shower}

As discussed above the MPI are simulated in \texttt{Pythia} following a decreasing $p_T$ evolution.
So we have that the total probability is given from the composition of the various contributions:
\begin{align}
	\frac{d\mathcal{P}}{dp_T}=&\left( \frac{d\mathcal{P}_{MPI}}{dp_T}+\displaystyle\sum\frac{d\mathcal{P}_{ISR}}{dp_T}+\displaystyle\sum\frac{d\mathcal{P}_{FSR}}{dp_T} \right) \nonumber \\ 
	&\times\exp\left\{ -\displaystyle\int_{p_T}^{p_T\,i-1} \left( \frac{d\mathcal{P}_{MPI}}{dp_T'}+\displaystyle\sum\frac{d\mathcal{P}_{ISR}}{dp_T'}+\displaystyle\sum\frac{d\mathcal{P}_{FSR}}{dp_T'} \right)\,dp_T' \right\}
\end{align}

\begin{figure}[!ht]
	\centering
	\includegraphics[width=12cm]{{img/PartonShower.png}}
	\caption{add caption}
	\label{fig:PartonShower}
\end{figure}

In \figRef{fig:PartonShower} are shown 4 parton-parton interactions with their associated showers (ISR and FSR). The downward evolution correspond to read the graph from top to bottom. The first hard interaction occur at a scale $p_T=p_{T1}$ while the following ones at lower scales $p_{T2}$, $p_{T3}$, $p_{T4}$. Each interaction is associated with is radiation the first one occurs at $p_T=p_{T1}'$.
The scatterings that occur at $p_{T2}$ and $p_{T3}$ are originating from the same mother parton. 

This diagram is related to one of the two hadrons. The full event can be illustrated if a similar diagram is drawn for the other hadron and connected to the full black circles.

\subsection{Beam Beam Remnants and primordial $k_T$}

What is left after that the evolution is end? the evolution in $p_T$ can create an arbitrary complicate final state. 

This final state contains contributions from the scattered and unscattered partons that don't enter the $p_T$ evolution. The last are the so called Beam Beam Remnants (BBR). 
BBR take into account the number of valence quark remaining and the number of sea quark required for the overall flavour conservation.

To ensure momentum conservation the BBR have to take all the remaining longitudinal momentum that is not extracted by the MPI initiators.

\subsubsection*{Primordial $k_T$}

We have considered only the longitudinal momentum. in a real scenario partons inside the hadrons are fermions so are expected to have a non zero initial transverse momentum arising from the Fermi motion inside the incoming hadrons. This is denoted as "Primordial $k_T$" this is different from the transverse momentum derived from DGLAP shower evolution or from the hard interaction.

\bigskip

Based on Fermi motion alone, one would expect a value of the primordial $k_T$ is estimate as: 
\begin{equation}
	k_T\simeq\frac{\hbar}{r_p}\approx\frac{0.2\ \mathrm{GeV\cdot fm}}{0.7\ \mathrm{fm}}\approx0.3\ \mathrm{GeV}
\label{eq:PrimordialKT}
\end{equation}
but for example to reproduce the data for the the $p_T$ distributions of $Z$ bosons produced in hadron–hadron collisions, one need a larger contribution. This phenomenon has not a satisfactory explanation. Until an explanation is found the idea is to consider a effective primordial $k_T$ for the initiators larger than the one in \eqRef{eq:PrimordialKT}.

In \texttt{Pythia} the primordial kT is assigned to initiators sampling a Gaussian distributions for $p_x$ and $p_y$ independently with variable width $\sigma$

\begin{equation}
	\sigma(Q,\widehat{m})=\frac{Q_{1/2}\,\sigma_{\text{soft}}+Q\,\sigma_{\text{soft}}}{Q_{1/2}+Q}\,\frac{\widehat{m}}{\widehat{m}_{1/2}+\widehat{m}}
\end{equation}
 
Where $Q$ is the hard-process renormalization scale for the main hard process and the $p_T$ scale for subsequent MPI. $\sigma_{\text{soft}}$, $\sigma_{\text{hard}}$ are the minimum and maximum width, $\widehat{m}$ is the invariant mass, while $Q_{1/2}$ and $\widehat{m}_{1/2}$ the halfway values between the two extremes.

\subsection{Color Reconnection and Hadronization }