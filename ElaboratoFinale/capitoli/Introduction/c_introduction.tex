%INTRODUCTION
%%%%-------PROVA-------------
%\documentclass[10pt]{report}
%\usepackage[british]{babel}
%\usepackage{amsmath}
%\usepackage{amsfonts}
%\usepackage{amssymb}
%\usepackage{geometry}
%\geometry{hmargin={2cm,2cm},vmargin={2cm,2cm}}
%\usepackage{biblatex} %Imports biblatex package
%\addbibresource{{../biblio/biblio.bib}} %Import the bibliography file
%%%%%-------------------------
%
%\begin{document}

\chapter{Introduction}

\setcounter{page}{1}
\pagenumbering{arabic}

	The \textbf{\emph{standard model of particle}} (SM) is the theory that describe all the elementary particles and the fundamental interactions: electromagnetic, weak and strong interactions\footnote{Actually the fundamental interactions are four: there is also the gravitational interaction. The inclusion of a description also for this last interaction is a central problem for the modern physics. There are different theory that try to merge the quantum effects with the general relativity that describe the gravitational force but none of these have been able to give a final description yet. This is not important in the framework of the particle physics studies at the collider in fact this force is negligible respect to the others.  }. The standard model relies on the Quantum Field Theory framework and on the concept of gauge symmetry. The symmetry group for the theory is $SU(3)\times SU(2)\times U(1)$. 
\\
The aim of the SM is to describe with a unique formalism all the three interactions. The abelian group $U(1)$ is related the electromagnetic interaction, the theory based on the gauge group $U(1)$, first introduced by Richard Feynman, is called \textit{quantum electrodynamics} (QED). During the last century this force was unified to the weak interaction described by the gauge symmetry $SU(2)$ in a single interaction: the Electroweak interaction \cite{PhysRevLett.19.1264}. Last, the $SU(3)$ group is the one related to the strong interactions of quarks the theory describe the interaction in term of three possible state of charge called \textit{colors}: red, green, blue. This is the origin of the theory name: \textit{quantum chromodynamics} (QCD). 
%This quantum field theory describe the forces acting between particles by means of the exchange of other particles representing the quanta of the interaction and called \textit{gauge bosons}. 
\\
The particle contained in the SM are grouped in \textit{fermions}, particles with semi-integer spin, and \textit{bosons}, particles with integer spin. The fermions contained in the standard model are sub-divided into leptons and quarks. The first one: $e,\,\nu_e,\,\mu,\,\nu_\mu,\,\tau,\,\nu_\tau$, can interact only by means of weak and electromagnetic interactions (not the neutrinos they have not electrical charge), in fact they have not a color charge so cannot interact strongly. While the quarks: $u,\,d,\,s,\,c,\,b,\,t$, can interact by means of all the above described interactions.  
While the boson are the gauge particles (spin 1), i.e. the particles that mediate the interaction: photons is the well known gauge boson for the electromagnetic force, gluons are the ones for the strong interaction, $W$ and $Z$ bosons are mediators of the weak one. Lastly, the famous Higgs boson that is the scalar field of the theory (particle with spin 0). The masses of all the particles in the SM are derived by the interaction with this last particle by mean of the Higgs mechanism. \tableRef{fig:particlesSM} shows a summary of all the fundamental particles of the SM with discovery dates, masses and some other properties.

\begin{figure}[!ht]
	\centering
	\includegraphics[width=0.95\textwidth]{{Particles.png}}
	\caption{All the fundamental particles of the standard model. Figure from \cite{CottinBuracchio:2276806}}
	\label{fig:particlesSM}
\end{figure}

\noindent The discover of the Higgs boson in 2012 \cite{HiggsBoson_discovery} at the Large Hadron Collider (\textsc{cern}) was the last ingredient that was missing in the SM. The Higgs boson was predicted by the theory but not yet discovered. This discovery confirm the theory and complete it. 
\\
In a high energy hadron-hadron collision all the interactions named above take place and participate in the creation of the complex final state observed in the experiments that investigate the fundamental particle physics. Obviously to explain what is observed in colliders experiments a good understanding of the theory that describes the strong interaction is required  

\medskip

Historically, the strong interaction was associated to the hadrons and not to quarks (they were unknown yet). However, with the deep inelastic scattering experiments \cite{DIS} it was proven that the hadrons are composite objects and then they cannot be fundamental particles for the SM. There must be something more fundamental: \textit{partons} (quarks and gluons). The \textit{partons model} \cite{Feynman:1969wa} was developed to describe the interaction of hadrons in very high energy scatterings. What was observed in DIS of electrons on nucleons was the so called \textit{Bjorken scaling} the observed structure constant was (at high energy) independent from the probe energy this reflect the fact that the nucleons, or more in general hadrons, are constituted by a collection of point-like objects: the above mentioned partons.
\\
So the partons model leads to the introduction of the \textit{Parton Distribution Function} (PDF) in order to describe the probability of finding a parton inside an hadron carrying a fraction of the total momentum of the hadron. 

\medskip

But why the quarks have not been directly observed by these high energy experiments? the reason why they have not been measured is that they are confined by the strong force into the hadrons. In fact, the intensity of the strong force increase when one try to take apart the quarks bounded together. 
\\
This peculiarity of the QCD theory, respect to the QED, arises from the fact that QCD group $SU(3)$ is not an abelian group this reflects in the color charge of the vector boson related to the strong interaction. In fact, the gluons are colored, so can interact each other. While in the QED this was not possible in fact the photons are not charged particles, and cannot interacts themselves. This peculiarity leads to an opposite behavior of the QCD coupling constant, $\alpha_s$, respect to the QED one, $\alpha$.   
\\
The evolution of the strong interaction coupling constant with respect to the momentum scale $Q^2$ of the interaction is shown in \figRef{fig:alphas}. When $Q^2\gg 1 \ \mathrm{GeV}$ (short distance) the $\alpha_s$ value is small and so the quark and the gluon are weakly coupled, this is called \textit{asymptotic freedom}. In this limit the processes can be evaluate using perturbation theory. This type of processes are refered as \textit{hard QCD} processes, where the hard attribute is related to the high energy scale of the interaction.   

\begin{figure}[!htb]
	\centering
	\includegraphics[width=0.55\textwidth]{{alphaS2.png}}
	\caption{evolution of the coupling constant with the energy scale of the interaction \cite{PDG}}
	\label{fig:alphas}
\end{figure}

\noindent On the other hand, when $Q^2$ is small (large distance), the value of the coupling constant is large, this leads to \textit{quarks confinement}. Color confinement is the  reason why we cannot observe free quarks in the experiments, as described above, but as the energy scale of the interaction decrease the quarks are confined into hadrons. These type of processes where the energy scale is low are referred as \textit{soft QCD} processes. Soft QCD processes cannot be described by perturbation theory\footnote{Due to the high value of the constant $\alpha_s$ the series expansion in terms of $\alpha_s$ is not possible.} and are not well understood starting from fundamental principles. 
\\
The soft QCD processes are clearly an important fraction of the processes we have to describe in order to understand what is going on in real proton-proton collision at LHC. 
%A clear example of the importance of the soft processes is the hadronization: once the partons inside the protons have interact the products of the interaction go toward the detector after a tiny distance all the quarks are bounded in hadrons. The hadronization is present in every collision, it is clear that we have to take it into account.  
%So, the use of Monte Carlo simulations to describe these processes is required to explain the experimental data we get. 
Every physics study at the LHC cannot ignore all these contribution in order to understand all features of main QCD processes i.e. minimum-bias events and jet production. This understanding requires not only a good description of the hard scatters but also a good description of the \textit{underlying event} (UE) i.e. the soft component of a collision that increase the complexity of the final state observed adding more particles to it.  

\medskip

To study the underlying event, or more in general soft QCD processes, it is required the use of Monte Carlo (MC) generators that often are based on phenomenological models with lots of free parameters. These free parameters must be tuned on experimental data in order to obtain meaningful results from the simulations. The procedure to estimate the best parameters values in order to best reproduce the observables under analysis is called \textit{tune}. 

The tuning procedure can be really computational expensive, due to the high time required to run a Monte Carlo generator, and to perform a tune it is required to run a generator a very large amount of times. So, to reduce the number of runs required the procedure usually taken for the tuning is a \textit{parameterization-based approach}. The approach consist in: fitting the histograms that are generated from these Monte Carlo simulations with different values for the input parameters under analysis, and study the variation defining a function that describe the bins behavior to the variation of the parameters. We are going to refers to this procedure as \textit{fitting the generator response}.

\medskip

This thesis will focus on this \textit{tuning problem} for the underlying event using an approach based on machine learning implement by means of Feed Forward Neural Networks (FFNNs). 

%%%%%% Riassunto Tesi
%%%%%% Da Controllare una volta finita!!

------------------------ Da Ricontrollare una volta finita ------------------------


%In this thesis we are going to focus on this aspect the Monte Carlo simulation have to describe the real data very well if we want to use it in experiments. The procedure to adapt the generator parameters values in order to reproduce the data correctly goes under the name of \textit{tune}. 
The next chapter is going to introduce the theoretical aspect in the description of an hadron-hadron collision. We start from the description of hard event using perturbation theory and than we move on the description of the all-order approaches (e.g. parton showers implemented in Monte Carlo generator). Then, we focus on the main processes described by the Monte Carlo generators and in particular in \textsc{pythia8}. Some particular attention is give to the Multi Parton Interaction, the Color Reconnection, the Primordial $k_T$, the Initial and Final State Radiations description. These are the main contribution we are focusing on in the tuning of the underlying event and minimum bias observables and of the $Z$ boson production cross section at low transverse momentum of the $Z$ boson. All the observables we are going to use in the tunes are described in \chapRef{chap:ObservabletoStudytheUnderlyingEvent} with an introduction on the existing CMS tunes called CP and a number from 1 to 5 that are the starting point for our work. The \chapRef{chap:TuneprocedureCP5TuneandMCNNTUNES} describe the basic concepts in the tune procedure in particular on the parametrization-based approach used for the tuning, in this chapter we also discuss the main tool we are going to use that is called \textsc{mcnntunes} it implement a tuning method based on machine learning. Our tunes for the underlying event and for the $Z$ boson spectrum are described respectively in \chapRef{chap:OurTunefortheUnderlyingEvent} and \chapRef{chap:primordialkTtune}.
 