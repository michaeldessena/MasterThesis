\chapter*{Conclusions}
\addcontentsline{toc}{chapter}{Conclusions}

In this work it has been validate \textsc{mcnntunes} as a valid tool for the tuning of the high energy physics Monte Carlo generator \textsc{pythia8}. Firstly, in order to validate it, the underlying event was tuned trying to reproduce the results obtained with CP5. The work started with a test phase with the variation of only two parameters: \texttt{MultipartonInteractions:}\-\texttt{pT0Ref} and \texttt{MultipartonInteractions:}\-\texttt{ecmPow}. Both the \textsc{mcnntunes} models perform very well in this test phase. 
\\
After, this test the number of parameters was extended to the same parameters tuned in CP5. So, to the two parameters already tuned have been added the parameters \texttt{MultipartonInteractions:}\-\texttt{core}\-\texttt{Radius}, \texttt{MultipartonInteractions:}\-\texttt{core}\-\texttt{Fraction} and \texttt{ColorReconnection:}\-\texttt{range}. In this phase we obtain very good results from the PerBin model. It seems to describe data also better then CP5 but not in all the distributions, in particular the distributions with larger experimental uncertainties. So, it was decided to include a re-weight for some bins. This leads to overall results more similar to the CP5 ones.
\\
On the other hands the Inverse model gives very bad results in this case it is not able to produce some consistent results.
This can be related to the complexity operation that is performed by this model: learn the function that describe the generator respond and try to invert it. This operation is very complex and this complexity increase a lot with the number of parameters.

Once the tool has been validated it is used to improve the description of the distributions of $Z$ boson production events in the low transverse momentum region. To do that the effect on these distribution of the parameters \texttt{BeamRemnants:}\-\texttt{primordialKT}\-\texttt{hard} and \texttt{SpaceShower:}\-\texttt{pT0Ref} was investigated. In fact, the CP5 tune does not describe these distribution very well, the values for these two parameters were inherited by the older Monash Tune. 
The new tune presented here for these two parameters was performed with both the models offered by \textsc{mcnntunes} and the results obtained are compatible each other. 

In the future works also the impact of the MPI has be investigated. In the last chapter it was showed that the effect of MPI is not negligible in the low transverse momentum region.
\\
Overall \textsc{mcnntunes} can be considered a valid tool for future tunes and it can become a cross check to the already existing and consolidate tool \textsc{professor}. Moreover, if some more attention is given to the Inverse model it can become a new powerful approach to the tuning procedure.
 