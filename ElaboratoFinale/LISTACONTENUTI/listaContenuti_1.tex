\documentclass[10pt]{article}
\usepackage[british]{babel}
\usepackage{amsmath}
\usepackage{amsfonts}
\usepackage{amssymb}
\usepackage{geometry}
\geometry{hmargin={2cm,2cm},vmargin={2cm,2cm}}
\usepackage{enumitem} % to change the enumerate label

\begin{document}

\section*{CONTENTS}

\begin{enumerate}[label=\arabic*)]
\item \textbf{Introduction} 
	\begin{itemize}
		\item introduction on standard model of particle ($\mathtt{U(1)\times SU(2)\times SU(3)}$)
	\end{itemize}
\item \textbf{Proton-Proton scattering}
	\begin{itemize}
		\item Hadronic cross section 
		\item Partonic cross section
		\item Parton Distribution Function (PDF)
	\end{itemize}
\item \textbf{Multiparton Interaction \& Beam-Beam remnant and Pythia}
	\begin{itemize}
		\item Why need Multiparton interactions
		\item ISR-MPI-FSR in Pythia8
		\item What is left? Beam-Beam remnants
		\item Fermi-Motion $\Rightarrow$ Primordial kT
		\item Color recconection range
		\item After that? Hadronization\dots
	\end{itemize}
\item \textbf{Underlyng event in proton-proton scattering}
\item \textbf{Observable to study the underlying event and multiple Interaction}
\item \textbf{Neural Nwtwork \& MCNNTUNES}
\begin{itemize}
	\item What is tune?
	\item What is \textsc{mcnntunes}?
	\item Neural Network introduction
\end{itemize}
\item \textbf{CP5 and our tune}
\begin{itemize}
	\item Introduction to CP tunes
	\item Our work on Minimum Bias events
	\item Our tunes
\end{itemize}
\item \textbf{Primordial kT}
\begin{itemize}
	\item Introduction on PrimordialkT on pTZ observation
	\item Our result on Primordial kT in pTZ measurament 
\end{itemize}
\item \textbf{Conclusion}
\end{enumerate}






\end{document}